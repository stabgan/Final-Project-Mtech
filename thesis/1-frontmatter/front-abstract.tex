\chapter*{Abstract}
\label{chap:abstract}

This thesis presents a novel approach to improving rare disease coverage in automated ICD coding by leveraging synthetic clinical text generation, advanced knowledge integration, and efficient computational strategies. First, it addresses the fundamental challenge of long-tail data scarcity in real-world medical records, which leads to suboptimal prediction performance for underrepresented ICD codes. To mitigate this, large language models (LLMs) are employed to produce factually grounded and diverse discharge summaries that capture realistic patient comorbidities, guided by structured medical ontologies such as SNOMED CT and Orphanet. A rigorous, multi-phase validation pipeline – combining rule-based checks for medical plausibility, LLM-based fact verification to detect subtle contradictions, and ICD feedback loops to ensure label consistency – filters out inaccurate synthetic samples. These high-quality, balanced datasets are then used to train and refine multi-label ICD classification models, including transformer-based architectures (e.g., PLM-ICD) and synonym-aware networks. Emphasis is placed on rare disease codes, with the synthetic corpus boosting their representation and improving macro-F1 and recall metrics. Addressing hardware constraints, the methodology integrates quantization (QLoRA), gradient checkpointing, and knowledge distillation so that larger models can be fine-tuned on a moderate GPU setup without sacrificing accuracy. The research further tackles error analysis, bias detection, and explainability, employing methods like attention visualization and SHAP attributions to ensure transparent, trustworthy predictions. By uniting sophisticated data augmentation, knowledge-based validation, and memory-efficient training, this work seeks to significantly advance the scalability and coverage of automated ICD coding systems, particularly for rare disorders where annotation gaps currently limit clinical AI effectiveness.

