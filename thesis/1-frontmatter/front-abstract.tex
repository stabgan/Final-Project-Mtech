\chapter*{Abstract}
\begin{doublespacing}
This thesis proposes a data-centric framework to improve automated ICD coding, especially for rare diseases. We tackle the challenge of limited annotated data by synthesizing realistic discharge summaries via large language models, guided by ontologies like SNOMED CT. A multi-step validation procedure—combining rule-based checks, fact verification, and ICD-label feedback—helps filter out inaccurate synthetic samples. The resulting balanced dataset boosts classification performance on underrepresented codes, yielding gains in macro-F1 and recall.  

Our approach also addresses hardware limitations through methods such as quantization and gradient checkpointing, enabling the fine-tuning of large models on modest GPUs. Finally, we incorporate interpretability tools, including attention maps and SHAP analyses, to enhance transparency and bias detection. Results indicate that selective data augmentation, combined with robust quality controls and efficient training, significantly advances the coverage and reliability of automated ICD coding.
\end{doublespacing}
