%=========================================================
% File: front-glossary-abbreviations.tex
% Description: Glossary entries & abbreviations
%=========================================================

% GLOSSARY ENTRIES
\newglossaryentry{icd}{
    name        = {ICD},
    description = {International Classification of Diseases, a standardized coding system for diagnoses and procedures in healthcare},
}

\newglossaryentry{rare}{
    name        = {Rare Diseases},
    description = {Diseases or conditions that occur infrequently and often lack sufficient annotated data for robust AI modeling},
}

\newglossaryentry{clinicalnotes}{
    name        = {Clinical Notes},
    description = {Unstructured medical narratives (e.g., discharge summaries) that contain patients’ diagnosis, treatments, and findings},
}

\newglossaryentry{llm}{
    name        = {Large Language Model (LLM)},
    description = {A type of neural network trained on massive text corpora to understand and generate human-like language},
}

\newglossaryentry{snomed}{
    name        = {SNOMED CT},
    description = {Systematized Nomenclature of Medicine -- Clinical Terms, a widely used standardized healthcare terminology},
}

\newglossaryentry{orphanet}{
    name        = {Orphanet},
    description = {A European reference portal for information on rare diseases and orphan drugs, offering structured medical ontologies},
}

\newglossaryentry{plm-icd}{
    name        = {PLM-ICD},
    description = {A transformer-based ICD coding model leveraging large pre-trained language models for improved text classification},
}

\newacronym{mimic}{MIMIC}{Medical Information Mart for Intensive Care}
\newacronym{qlora}{QLoRA}{Quantization LoRA, a parameter-efficient technique for fine-tuning large models}
\newacronym{nlp}{NLP}{Natural Language Processing}
\newacronym{gpu}{GPU}{Graphics Processing Unit}
\newacronym{shap}{SHAP}{SHapley Additive exPlanations, a method for interpreting model outputs}

% Optionally print the glossary and acronyms here if your template calls for it
% e.g., \printglossaries or \printGlossaryAndAbbreviations
