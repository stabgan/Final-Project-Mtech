% This is chap-results.tex
\chapter{Results}

\section{Baseline Model Experiments}

\subsection{K-Nearest Neighbors (KNN) Classifier}
\begin{itemize}
    \item \textbf{Objective}: Establish a baseline for multi-label ICD code prediction using simple models.
    \item \textbf{Methodology}:
    \begin{itemize}
        \item \textbf{Text Vectorization}: Used TF-IDF vectorizer with a maximum of 5,000 features and n-gram range of (1,2).
        \item \textbf{Multi-Label Binarization}: Applied \textbf{MultiLabelBinarizer} to handle multiple ICD codes per clinical note.
        \item \textbf{Model Training}: Trained a KNN classifier on the vectorized text data.
    \end{itemize}
    \item \textbf{Results}:
    \begin{itemize}
        \item \textbf{Evaluation Metrics}: Used Hamming Loss and Macro F1-Score to assess performance.
        \item \textbf{Performance}: The model achieved a Macro F1-Score of 0.12, indicating poor performance, especially on rare codes.
        \item \textbf{Analysis}: High Hamming Loss and low F1-Score confirmed that the model struggled with data imbalance and text complexity.
    \end{itemize}
\end{itemize}

\subsection{Observations}
\begin{itemize}
    \item The baseline model highlighted the challenges in predicting ICD codes using traditional machine learning approaches.
    \item Reinforced the need for advanced models that can handle complex clinical texts and data imbalance.
\end{itemize}

\section{Similarity Matching Using MinHash}

\subsection{Objective}
Identify duplicate or highly similar clinical notes to reduce redundancy and potentially improve model training.

\subsection{Methodology}
\begin{itemize}
    \item \textbf{Text Preprocessing}: Applied standard preprocessing steps to clean the text.
    \item \textbf{MinHash and LSH}: Used MinHash to create signatures for each document and Locality Sensitive Hashing (LSH) to efficiently identify similar documents.
    \item \textbf{Threshold Setting}: Set similarity thresholds (e.g., 90\% and 97\%) to identify pairs of similar notes.
\end{itemize}

\subsection{Results}
\begin{itemize}
    \item Identified 260 notes with over 90\% similarity.
    \item Found six notes with over 97\% similarity, indicating nearly identical content.
    \item Verification with a medical expert confirmed that these notes were indeed similar or duplicates.
\end{itemize}

\subsection{Implications}
\begin{itemize}
    \item Removing or consolidating similar notes can reduce data redundancy.
    \item May improve model performance by preventing the model from being biased towards duplicated information.
\end{itemize}

\section{Challenges Identified}
\begin{itemize}
    \item \textbf{Data Imbalance}: The initial models confirmed that frequent codes dominate predictions, and rare codes are often missed.
    \item \textbf{Model Limitations}: Simple models like KNN are insufficient for capturing the complexity of clinical narratives.
    \item \textbf{Need for Advanced Techniques}: Emphasized the importance of implementing transformer-based models and data augmentation strategies.
\end{itemize}
