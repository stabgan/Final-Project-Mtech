\documentclass{beamer}

% Theme and color theme
\usetheme{Madrid}
\usecolortheme{dolphin}

% Packages
\usepackage{graphicx}
\usepackage{hyperref}
\usepackage{booktabs}
\usepackage{tikz}
\usepackage{pgfgantt}
\usepackage{pgfplots}
\usepackage{multicol}

% Title, author, etc.
\title[Hybrid Data Augmentation and xAI]{Hybrid Data Augmentation and Explainable AI for ICD Code Prediction}
\author[Kaustabh Ganguly]{Kaustabh Ganguly \\ Roll Number: CH23M514}
\institute{}
\date{September 22, 2024}

\begin{document}

% Title slide
\begin{frame}
    \titlepage
\end{frame}

% Slide 1: Project Summary
\begin{frame}{Project Summary}
    \begin{itemize}
        \item \textbf{Problem:} Manual ICD code annotation is labor-intensive and error-prone.
        \item \textbf{Goal:} Accurately predict ICD codes from clinical narratives, focusing on rare codes.
        \item \textbf{Approach:}
        \begin{itemize}
            \item Extract medical entities from clinical text.
            \item Map entities to ontologies like SNOMED CT and RxNorm.
            \item Use transformer-based models for embeddings.
        \end{itemize}
    \end{itemize}
\end{frame}

% Slide 2: Project Summary (continued)
\begin{frame}{Project Summary (continued)}
    \begin{itemize}
        \item \textbf{Hybrid Data Augmentation Strategy:}
        \begin{itemize}
            \item Combines Retrieval Augmented Generation (RAG) with ontology-based techniques.
            \item Enrich ICD code descriptions using external sources (PubMed, Wikipedia, WHO API).
        \end{itemize}
        \item \textbf{Generate Synthetic Clinical Notes:}
        \begin{itemize}
            \item Use generative models (GANs) for rare ICD codes.
            \item Incorporate entities from RxNorm, SNOMED CT, and other ontologies.
        \end{itemize}
        \item \textbf{Leverage Knowledge Graphs:}
        \begin{itemize}
            \item Construct knowledge graphs from medical ontologies.
            \item Capture complex relationships between medical entities.
        \end{itemize}
    \end{itemize}
\end{frame}

% Slide 3: Objectives (1)
\begin{frame}{Objectives}
    \begin{enumerate}
        \item Develop transformer-based model mapping clinical text embeddings to enriched ICD code embeddings.
        \item Implement hybrid data augmentation combining RAG with ontology-based methods.
        \item Leverage knowledge graphs and ontologies (SNOMED CT, RxNorm, ICD-10-CM).
        \item Employ advanced classification techniques:
        \begin{itemize}
            \item Label embedding
            \item Hierarchical classification
            \item Label grouping
        \end{itemize}
    \end{enumerate}
\end{frame}

% Slide 4: Objectives (continued)
\begin{frame}{Objectives (continued)}
    \begin{enumerate}
        \setcounter{enumi}{4}
        \item Integrate explainability techniques:
        \begin{itemize}
            \item SHAP
            \item LIME
            \item Integrated Gradients
        \end{itemize}
        \item Achieve significant improvement in evaluation metrics (e.g., 10\% increase in macro F2-score on rare ICD codes).
        \item Develop open-source toolkit automating ICD coding process.
    \end{enumerate}
\end{frame}

% Slide 5: Literature Survey
\begin{frame}{Literature Survey}
    \begin{itemize}
        \item \textbf{Transformer-based models} (BERT, BioBERT, ClinicalBERT) improved ICD coding.
        \item \textbf{Challenges} remain in predicting rare ICD codes due to data scarcity.
        \item \textbf{Data augmentation techniques} explored:
        \begin{itemize}
            \item GANs for synthetic clinical notes (Wang et al.)
            \item Retrieval Augmented Generation (RAG)
        \end{itemize}
        \item \textbf{Integration of knowledge graphs} from ontologies enhances models.
        \item \textbf{Explainable AI (XAI)} techniques (SHAP, LIME, Integrated Gradients) important for clinician trust.
        \item \textbf{Gaps} in combining RAG, knowledge graphs, and XAI.
        \item Our project aims to \textbf{bridge these gaps}.
    \end{itemize}
\end{frame}

% Slide 6: Datasets
\begin{frame}{Datasets}
    \begin{itemize}
        \item \textbf{Primary Datasets:}
        \begin{itemize}
            \item \textbf{MIMIC-IV Clinical Database}
            \begin{itemize}
                \item Over 40,000 patient records with clinical notes linked to ICD codes.
            \end{itemize}
            \item \textbf{eICU Collaborative Research Database}
            \item \textbf{Jon Snow Labs Synthetic Data}
        \end{itemize}
        \item \textbf{Additional Resources:}
        \begin{itemize}
            \item WHO API for ICD-10-CM codes and information.
            \item External medical literature (PubMed, Wikipedia).
        \end{itemize}
        \item \textbf{Ethical Considerations:}
        \begin{itemize}
            \item Compliance with data use agreements.
            \item Ensuring patient privacy and data security.
        \end{itemize}
    \end{itemize}
\end{frame}

% Slide 7: Workplan
\begin{frame}{Workplan}
    \begin{itemize}
        \item \textbf{Phase 1: Data Preparation (Months 1-3)}
        \begin{itemize}
            \item Data acquisition and preprocessing
            \item Entity extraction and ontology mapping
            \item Knowledge graph construction
        \end{itemize}
        \item \textbf{Phase 2: Model Development and Evaluation (Months 4-7)}
        \begin{itemize}
            \item Hybrid data augmentation
            \item Model architecture design
            \item Multi-label classification handling
            \item Model training and validation
            \item Performance evaluation
        \end{itemize}
        \item \textbf{Phase 3: Explainability Integration (Months 7-9)}
        \begin{itemize}
            \item Integrate explainability tools
            \item Develop visualizations
        \end{itemize}
        \item \textbf{Phase 4: Deployment \& Dissemination (Months 8-9)}
        \begin{itemize}
            \item Toolkit development
            \item Documentation
            \item Final reporting
        \end{itemize}
        \item \textbf{Phase 5: Human-in-the-Loop System (Months 10-11)}
        \begin{itemize}
            \item Interface for clinicians
            \item Framework for feedback integration
        \end{itemize}
    \end{itemize}
\end{frame}

% Slide 8: Expected Deliverables and Implementation Arrangements
\begin{frame}{Expected Deliverables and Implementation Arrangements}
    \begin{itemize}
        \item \textbf{Expected Deliverables:}
        \begin{itemize}
            \item Advanced, explainable ICD code prediction model.
            \item Enriched ICD code embeddings.
            \item Open-source toolkit (data pipelines, model code, synthetic data generation).
            \item Detailed experimental records.
            \item Enhanced model interpretability for clinician trust.
            \item Framework for human-in-the-loop integration.
        \end{itemize}
        \item \textbf{Implementation Arrangements:}
        \begin{itemize}
            \item Guided by mentors Samyabrata Chakraborty and Debopam Nanda.
            \item Collaboration with clinicians for domain expertise.
            \item Regular bi-weekly meetings with healthcare professionals.
            \item Agile methodology with two-week sprints.
        \end{itemize}
    \end{itemize}
\end{frame}

% Slide 9: Resource Requirements
\begin{frame}{Resource Requirements}
    \begin{itemize}
        \item \textbf{Hardware:}
        \begin{itemize}
            \item Google Colab Pro for initial experiments.
            \item AWS/GCP credits for scalable training.
            \item Local machine: 32GB RAM, NVIDIA RTX 3060 GPU, 2TB storage.
        \end{itemize}
        \item \textbf{Software:}
        \begin{itemize}
            \item Open-source tools: Python, PyTorch/TensorFlow, Hugging Face Transformers.
            \item Explainability tools: SHAP, LIME.
            \item Entity linking tools: cTAKES, MedSpaCy, MedCat.
        \end{itemize}
        \item \textbf{Data Access:}
        \begin{itemize}
            \item MIMIC-IV database, eICU Collaborative Research Database.
            \item Medical ontologies: SNOMED CT, RxNorm, ICD-10-CM (WHO API).
        \end{itemize}
    \end{itemize}
\end{frame}

% Slide 10: Thank You
\begin{frame}
    \centering
    \Huge{Thank You!}
\end{frame}

\end{document}
